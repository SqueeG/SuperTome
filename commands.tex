%%%%%%%%%%%%%%%%%%%%%%%%%%%%%%%%%%%%%%%%%%%%%%%%%%
%%%%%%%%%%%%%%%%%%%%%%%%%%%%%%%%%%%%%%%%%%%%%%%%%%
%%% Revised Commands
%%%%%%%%%%%%%%%%%%%%%%%%%%%%%%%%%%%%%%%%%%%%%%%%%%
%%%%%%%%%%%%%%%%%%%%%%%%%%%%%%%%%%%%%%%%%%%%%%%%%%
\makeatletter

%fiddles with how chapter titles are displayed
\renewcommand{\@makechapterhead}[1]{%
\vspace*{0 pt}{%
\raggedright \normalfont \fontsize{32}{32} \selectfont \bfseries%
\ifnum \value{secnumdepth}>-1%
  \if@mainmatter \vspace{-8pt} {\fontsize{20}{20} \selectfont Chapter \thechapter:}\\[8pt]%
  \fi%
\fi
\hspace{0.65cm} #1\par\nobreak\vspace{20 pt}%
}}

%makes paragraphs show up closer together
\renewcommand{\paragraph}{%
\@startsection{paragraph}{4}%
{\z@}{1.0ex \@plus 1ex \@minus 0.2ex}{-1em} % wtf is an 'ex' anyways?
{\normalfont\normalsize\bfseries}%
}

%lets multicolumn have the proper background colors as defined by rowcolors
\let\oldmc\multicolumn
\newcommand{\mcinherit}{% Activate \multicolumn inheritance
  \renewcommand{\multicolumn}[3]{%
    \oldmc{##1}{##2}{\ifodd\rownum \@oddrowcolor\else\@evenrowcolor\fi ##3}%
  }}

\makeatother

%add labels within sections, subsections, and subsubsections
\LetLtxMacro{\oldsection}{\section}
\renewcommand{\section}[1]{\oldsection{#1}\label{sec:#1}}

\LetLtxMacro{\oldsubsection}{\subsection}
\renewcommand{\subsection}[1]{\oldsubsection{#1}\label{sec:#1}}

\LetLtxMacro{\oldsubsubsection}{\subsubsection}
\renewcommand{\subsubsection}[1]{\oldsubsubsection{#1}\label{sec:#1}}

%only put chapters and sections into the TOC
\setcounter{secnumdepth}{1}
%makes a subsubsection start off indented.
\setlength{\beforesubsubsecskip}{-\beforesubsubsecskip}

%%%%%%%%%%%%%%%%%%%%%%%%%%%%%%%%%%%%%%%%%%%%%%%%%%
%%%%%%%%%%%%%%%%%%%%%%%%%%%%%%%%%%%%%%%%%%%%%%%%%%
%%% General Commands
%%%%%%%%%%%%%%%%%%%%%%%%%%%%%%%%%%%%%%%%%%%%%%%%%%
%%%%%%%%%%%%%%%%%%%%%%%%%%%%%%%%%%%%%%%%%%%%%%%%%%
\newcommand{\todo}[1]{\index{todo!#1}}

\newcommand{\tagline}[1]{\vspace{-6pt} \textit{#1} \medskip}

\newcommand{\gameterm}[1]{#1\index{#1}}

\newcommand{\raceentry}[1]{\oldsection{#1}\index{#1}\label{race:#1}}

\newcommand{\skillentry}[2]{\oldsubsection[#1]{#1 #2}\index{#1 (skill)}\label{skill:#1}}

\NewDocumentCommand\featentry{m+g}{%
  \IfNoValueTF{#2}
    {\oldsubsubsection[#1]{#1 [General]}\label{feat:#1}}%no second arg, general feat
    {\oldsubsubsection[#1]{#1 #2}\label{feat:#1}}%second arg, special type of feat
}

\newcommand{\spellentry}[1]{\oldsubsubsection{#1}\label{spell:#1}}

\NewDocumentCommand\linkrace{m+g}{%
  \IfNoValueTF{#2}
    {\hyperref[race:#1]{#1}}%no second arg, display is same as link
    {\hyperref[race:#1]{#2}}%second arg, link to first and display second
}
\NewDocumentCommand\linkclass{m+g}{%
  \IfNoValueTF{#2}
    {\hyperref[class:#1]{#1}}%no second arg, display is same as link
    {\hyperref[class:#1]{#2}}%second arg, link to first and display second
}
\NewDocumentCommand\linkskill{m+g}{%
  \IfNoValueTF{#2}
    {\hyperref[skill:#1]{#1}}%no second arg, display is same as link
    {\hyperref[skill:#1]{#2}}%second arg, link to first and display second
}
\NewDocumentCommand\linkfeat{m+g}{%
  \IfNoValueTF{#2}
    {\hyperref[feat:#1]{#1}}%no second arg, display is same as link
    {\hyperref[feat:#1]{#2}}%second arg, link to first and display second
}
\NewDocumentCommand\linkspell{m+g}{%
  \IfNoValueTF{#2}
    {\hyperref[spell:#1]{#1}}%no second arg, display is same as link
    {\hyperref[spell:#1]{#2}}%second arg, link to first and display second
}
\NewDocumentCommand\linkcondition{m+g}{%
  \IfNoValueTF{#2}
    {\hyperref[condition:#1]{#1}}%no second arg, display is same as link
    {\hyperref[condition:#1]{#2}}%second arg, link to first and display second
}
\NewDocumentCommand\linkability{m+g}{%
  \IfNoValueTF{#2}
    {\hyperref[ability:#1]{#1}}%no second arg, display is same as link
    {\hyperref[ability:#1]{#2}}%second arg, link to first and display second
}
\NewDocumentCommand\linksec{m+g}{%
  \IfNoValueTF{#2}
    {\hyperref[sec:#1]{#1}}%no second arg, display is same as link
    {\hyperref[sec:#1]{#2}}%second arg, link to first and display second
}

\newcommand{\testempty}{\empty}
\newcommand{\isempty}{\empty}
%Two commands that can be compared to one another for \ifx logic tests.
%\isempty should never be changed.
%If \testempty holds a value of anything but empty, the test should return false.

\newenvironment{basictable}[2]
{
\begin{table}[htb]
\rowcolors{1}{white}{offyellow}
\caption{#1}
\centering
\begin{tabular}{#2}
}{
\end{tabular}
\end{table}
}

%%%%%%%%%%%%%%%%%%%%%%%%%%%%%%%%%%%%%%%%%%%%%%%%%%
%%%%%%%%%%%%%%%%%%%%%%%%%%%%%%%%%%%%%%%%%%%%%%%%%%
%%% Class Chapter Special Commands
%%%%%%%%%%%%%%%%%%%%%%%%%%%%%%%%%%%%%%%%%%%%%%%%%%
%%%%%%%%%%%%%%%%%%%%%%%%%%%%%%%%%%%%%%%%%%%%%%%%%%

\newcommand{\currentclassname}{foo}
\newcommand{\classentry}[1]{\oldsection{#1}\index{#1}\label{class:#1}\renewcommand{\currentclassname}{#1}}
%\newcommand{\Requirements}{\oldsubsubsection*{Requirements}}
%\newcommand{\Basics}{\oldsubsubsection*{Basics}}
\newcommand{\classfeatures}{\oldsubsubsection*{Class Features}
All of the following are class features of the \currentclassname}

%%%%Class Table Commands

\newcommand{\gbab}{\empty}
\newcommand{\mbab}{\empty}
\newcommand{\fort}{\empty}
\newcommand{\refl}{\empty}
\newcommand{\will}{\empty}
%Creates new commands for use in \ifx statements for formatting purposes.

\newcommand{\goodbab}{\renewcommand{\gbab}{\empty}\renewcommand{\mbab}{a}}
\newcommand{\modebab}{\renewcommand{\gbab}{a}\renewcommand{\mbab}{\empty}}
\newcommand{\poorbab}{\renewcommand{\gbab}{a}\renewcommand{\mbab}{a}}
%A set of commands to tell LaTeX what BAB progression the class has. Only one should be called per class.

\newcommand{\goodfor}{\renewcommand{\fort}{\empty}}
\newcommand{\poorfor}{\renewcommand{\fort}{a}}
%A set of commands to tell LaTeX what Fortitude progression the class has. Only one should be called per class.

\newcommand{\goodref}{\renewcommand{\refl}{\empty}}
\newcommand{\poorref}{\renewcommand{\refl}{a}}
%A set of commands to tell LaTeX what Reflex progression the class has. Only one should be called per class.

\newcommand{\goodwil}{\renewcommand{\will}{\empty}}
\newcommand{\poorwil}{\renewcommand{\will}{a}}
%A set of commands to tell LaTeX what Will progression the class has. Only one should be called per class.

\newenvironment{classtable}
{
\begin{table}[htb]
\rowcolors{1}{white}{offyellow}
\caption{The \currentclassname}
\small
\centering
\begin{tabular}{l l *{3}{c} l}
\textbf{Level} & \textbf{BAB} & \textbf{Fort} & \textbf{Reflex} & \textbf{Will} & \textbf{Special}\\
}{
\end{tabular}
\end{table}
}
%A a new environment that sets up the class tables. Include the \level commands between \begin{classtable}.

\newenvironment{extraclasstable}[1]
{
\begin{table}[htb]
\rowcolors{1}{white}{offyellow}
\caption{The \currentclassname}
\small
\centering
\begin{tabular}{l l *{3}{c} l c}
\textbf{Level} & \textbf{BAB} & \textbf{Fort} & \textbf{Reflex} & \textbf{Will} & \textbf{Special} & #1\\
}{
\end{tabular}
\end{table}
}
%A a new environment that sets up the class tables. Include the \level commands between \begin{classtable}

\newenvironment{minorcastingclasstable}
{
\begin{table}[htb]
\rowcolors{1}{offyellow}{white}
\caption{The \currentclassname}
\small
\centering
\begin{tabular}{l l *{3}{c} l *{7}{c}}
\multicolumn{6}{c}{\cellcolor{white}} & \multicolumn{7}{c}{\cellcolor{white} \textbf{Spells Per Day}}\\
\textbf{Level} & \textbf{BAB} & \textbf{Fort} & \textbf{Ref} & \textbf{Will} & \textbf{Special} & \textbf{0}&\textbf{1}&\textbf{2}&\textbf{3}&\textbf{4}&\textbf{5}&\textbf{6}\\
}{
\end{tabular}
\end{table}
}
%A a new environment similar to classtable, but with columns for a minor (zero through six) spell slot progression.
%uses "reverse" rowcolors from the base classtable because of the extra row that says "spells per day"

\newenvironment{fullcastingclasstable}
{
\begin{table}[htb]
\rowcolors{1}{offyellow}{white}
\caption{The \currentclassname}
\small
\centering
\begin{tabular}{l l c c c l *{10}{c}}
\multicolumn{6}{c}{\cellcolor{white}} & \multicolumn{10}{c}{\cellcolor{white}\textbf{Spells Per Day}}\\
\textbf{Level} & \textbf{BAB} & \textbf{Fort} & \textbf{Ref} & \textbf{Wil} & \textbf{Special}& \textbf{0}&\textbf{1}&\textbf{2}&\textbf{3}&\textbf{4}&\textbf{5}&\textbf{6}&\textbf{7}&\textbf{8}&\textbf{9}\\
}{
\end{tabular}
\end{table}
}
%Another environment for class tables, this one for full (0 through 9) spell slot progression.
%uses "reverse" rowcolors from the base classtable because of the extra row that says "spells per day"

\newcommand{\levelone}[1]{
1st  & \ifx\gbab\isempty +1 \else\ifx\mbab\isempty +0 \else +0 \fi \fi
	 & \ifx\fort\isempty +2 \else +0 \fi
	 & \ifx\refl\isempty +2 \else +0 \fi
	 & \ifx\will\isempty +2 \else +0 \fi
	 & #1 \\}
%A command that declares a table row within the class feature table.

\newcommand{\leveltwo}[1]{
2nd  & \ifx\gbab\isempty +2 \else\ifx\mbab\isempty +1 \else +1 \fi \fi
	 & \ifx\fort\isempty +3 \else +0 \fi
	 & \ifx\refl\isempty +3 \else +0 \fi
	 & \ifx\will\isempty +3 \else +0 \fi
	 & #1 \\}

\newcommand{\levelthree}[1]{
3rd  & \ifx\gbab\isempty +3 \else\ifx\mbab\isempty +2 \else +1 \fi \fi
	 & \ifx\fort\isempty +3 \else +1 \fi
	 & \ifx\refl\isempty +3 \else +1 \fi
	 & \ifx\will\isempty +3 \else +1 \fi
	 & #1 \\}

\newcommand{\levelfour}[1]{
4th  & \ifx\gbab\isempty +4 \else\ifx\mbab\isempty +3 \else +2 \fi \fi
	 & \ifx\fort\isempty +4 \else +1 \fi
	 & \ifx\refl\isempty +4 \else +1 \fi
	 & \ifx\will\isempty +4 \else +1 \fi
	 & #1 \\}

\newcommand{\levelfive}[1]{
5th  & \ifx\gbab\isempty +5 \else\ifx\mbab\isempty +3 \else +2 \fi \fi
	 & \ifx\fort\isempty +4 \else +1 \fi
	 & \ifx\refl\isempty +4 \else +1 \fi
	 & \ifx\will\isempty +4 \else +1 \fi
	 & #1 \\}

\newcommand{\levelsix}[1]{
6th  & \ifx\gbab\isempty +6 \else\ifx\mbab\isempty +4 \else +3 \fi \fi
	 & \ifx\fort\isempty +5 \else +2 \fi
	 & \ifx\refl\isempty +5 \else +2 \fi
	 & \ifx\will\isempty +5 \else +2 \fi
	 & #1 \\}

\newcommand{\levelseven}[1]{
7th  & \ifx\gbab\isempty +7 \else\ifx\mbab\isempty +5 \else +3 \fi \fi
	 & \ifx\fort\isempty +5 \else +2 \fi
	 & \ifx\refl\isempty +5 \else +2 \fi
	 & \ifx\will\isempty +5 \else +2 \fi
	 & #1 \\}

\newcommand{\leveleight}[1]{
8th  & \ifx\gbab\isempty +8 \else\ifx\mbab\isempty +6 \else +4 \fi \fi
	 & \ifx\fort\isempty +6 \else +2 \fi
	 & \ifx\refl\isempty +6 \else +2 \fi
	 & \ifx\will\isempty +6 \else +2 \fi
	 & #1 \\}

\newcommand{\levelnine}[1]{
9th  & \ifx\gbab\isempty +9 \else\ifx\mbab\isempty +6 \else +4 \fi \fi
	 & \ifx\fort\isempty +6 \else +3 \fi
	 & \ifx\refl\isempty +6 \else +3 \fi
	 & \ifx\will\isempty +6 \else +3 \fi
	 & #1 \\}

\newcommand{\levelten}[1]{
10th & \ifx\gbab\isempty +10 \else\ifx\mbab\isempty +7 \else +5 \fi \fi
	 & \ifx\fort\isempty +7 \else +3 \fi
	 & \ifx\refl\isempty +7 \else +3 \fi
	 & \ifx\will\isempty +7 \else +3 \fi
	 & #1 \\}

\newcommand{\leveleleven}[1]{
11th & \ifx\gbab\isempty +11 \else\ifx\mbab\isempty +8 \else +5 \fi \fi
	 & \ifx\fort\isempty +7 \else +3 \fi
	 & \ifx\refl\isempty +7 \else +3 \fi
	 & \ifx\will\isempty +7 \else +3 \fi
	 & #1 \\}

\newcommand{\leveltwelve}[1]{
12th & \ifx\gbab\isempty +12 \else\ifx\mbab\isempty +9 \else +6 \fi \fi
	 & \ifx\fort\isempty +8 \else +4 \fi
	 & \ifx\refl\isempty +8 \else +4 \fi
	 & \ifx\will\isempty +8 \else +4 \fi
	 & #1 \\}

\newcommand{\levelthirteen}[1]{
13th & \ifx\gbab\isempty +13 \else\ifx\mbab\isempty +9 \else +6 \fi \fi
	 & \ifx\fort\isempty +8 \else +4 \fi
	 & \ifx\refl\isempty +8 \else +4 \fi
	 & \ifx\will\isempty +8 \else +4 \fi
	 & #1 \\}

\newcommand{\levelfourteen}[1]{
14th & \ifx\gbab\isempty +14 \else\ifx\mbab\isempty +10 \else +7 \fi \fi
	 & \ifx\fort\isempty +9 \else +4 \fi
	 & \ifx\refl\isempty +9 \else +4 \fi
	 & \ifx\will\isempty +9 \else +4 \fi
	 & #1 \\}

\newcommand{\levelfifteen}[1]{
15th & \ifx\gbab\isempty +15 \else\ifx\mbab\isempty +11 \else +7 \fi \fi
	 & \ifx\fort\isempty +9 \else +5 \fi
	 & \ifx\refl\isempty +9 \else +5 \fi
	 & \ifx\will\isempty +9 \else +5 \fi
	 & #1 \\}

\newcommand{\levelsixteen}[1]{
16th & \ifx\gbab\isempty +16 \else\ifx\mbab\isempty +12 \else +8 \fi \fi
	 & \ifx\fort\isempty +10 \else +5 \fi
	 & \ifx\refl\isempty +10 \else +5 \fi
	 & \ifx\will\isempty +10 \else +5 \fi
	 & #1 \\}

\newcommand{\levelseventeen}[1]{
17th & \ifx\gbab\isempty +17 \else\ifx\mbab\isempty +12 \else +8 \fi \fi
	 & \ifx\fort\isempty +10 \else +5 \fi
	 & \ifx\refl\isempty +10 \else +5 \fi
	 & \ifx\will\isempty +10 \else +5 \fi
	 & #1 \\}

\newcommand{\leveleighteen}[1]{
18th & \ifx\gbab\isempty +18 \else\ifx\mbab\isempty +13 \else +9 \fi \fi
	 & \ifx\fort\isempty +11 \else +6 \fi
	 & \ifx\refl\isempty +11 \else +6 \fi
	 & \ifx\will\isempty +11 \else +6 \fi
	 & #1 \\}

\newcommand{\levelnineteen}[1]{
19th & \ifx\gbab\isempty +19 \else\ifx\mbab\isempty +14 \else +9 \fi \fi
	 & \ifx\fort\isempty +11 \else +6 \fi
	 & \ifx\refl\isempty +11 \else +6 \fi
	 & \ifx\will\isempty +11 \else +6 \fi
	 & #1 \\}

\newcommand{\leveltwenty}[1]{
20th & \ifx\gbab\isempty +20 \else\ifx\mbab\isempty +15 \else +10 \fi \fi
	 & \ifx\fort\isempty +12 \else +6 \fi
	 & \ifx\refl\isempty +12 \else +6 \fi
	 & \ifx\will\isempty +12 \else +6 \fi
	 & #1 \\}
