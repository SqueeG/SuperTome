%%%%%%%%%%%%%%%%%%%%%%%%%%%%%%%%%%%%%%%%%%%%%%%%%%
\classentry{Bard}
%%%%%%%%%%%%%%%%%%%%%%%%%%%%%%%%%%%%%%%%%%%%%%%%%%
\tagline{"Oh, really? I could write a song about that too\ldots{}"}

\textbf{Alignment:} Bards can be of any alignment. Some will argue that Bards can't be Lawful because it binds their free music spirit or whatever. However, while music is expressionistic, it is also mathematical. Already there are computers that can write music that is indistinguishable from the boring parts of Mozart in which he's just going up and down scales in order to mark time.

\textbf{Races:} Almost every race has its share of Bards, though there are slightly more found among the Elves than others.

\textbf{Starting Gold:} 3d6x10 gp (105 gold)

\textbf{Starting Age:} As Fighter.

\textbf{Hit Die:} d6

\textbf{Class Skills:} The bard's class skills (and the key ability for each skill) are \linkskill{Appraise} (Int), \linkskill{Balance} (Dex), \linkskill{Bluff} (Cha), \linkskill{Climb} (Str), \linkskill{Concentration} (Con), \linkskill{Craft} (Int), \linkskill{Decipher Script} (Int), \linkskill{Diplomacy} (Cha), \linkskill{Disguise} (Cha), \linkskill{Escape Artist} (Dex), \linkskill{Gather Information} (Cha), \linkskill{Hide} (Dex), \linkskill{Jump} (Str), \linkskill{Knowledge} (all skills, taken individually) (Int), \linkskill{Listen} (Wis), \linkskill{Move Silently} (Dex), \linkskill{Perform} (Cha), \linkskill{Profession} (Wis), \linkskill{Sense Motive} (Wis), \linkskill{Sleight of Hand} (Dex), \linkskill{Speak Language} (n/a), \linkskill{Spellcraft} (Int), \linkskill{Swim} (Str), \linkskill{Tumble} (Dex), and \linkskill{Use Magic Device} (Cha).

\textbf{Skills/Level:} 6 + Intelligence Bonus

\modebab{}
\poorfor{}
\goodref{}
\goodwil{}

\begin{minorcastingclasstable}
\levelone{\multicolumn{1}{p{4.8cm}}{\raggedright Bardic Music, Bardic Knowledge, Countersong, Fascinate, Inspire Courage +1} & 2 & -- & -- & -- & -- & -- & --}
\leveltwo{-- & 3 & 0 & -- & -- & -- & -- & --}
\levelthree{Inspire Competence & 3 & 1 & -- & -- & -- & -- & --}
\levelfour{-- & 3 & 2 & 0 & -- & -- & -- & --}
\levelfive{-- & 3 & 3 & 1 & -- & -- & -- & --}
\levelsix{Suggestion & 3 & 3 & 2 & -- & -- & -- & --}
\levelseven{-- & 3 & 3 & 2 & 0 & -- & -- & --}
\leveleight{Inspire Courage +2 & 3 & 3 & 3 & 1 & -- & -- & --}
\levelnine{Inspire Greatness & 3 & 3 & 3 & 2 & -- & -- & --}
\levelten{-- & 3 & 3 & 3 & 2 & 0 & -- & --}
\leveleleven{-- & 3 & 3 & 3 & 3 & 1 & -- & --}
\leveltwelve{Song of Freedom & 3 & 3 & 3 & 3 & 2 & -- & --}
\levelthirteen{-- & 3 & 3 & 3 & 3 & 2 & 0 & --}
\levelfourteen{Inspire Courage +3 & 4 & 3 & 3 & 3 & 3 & 1 & --}
\levelfifteen{Inspire Heroics & 4 & 4 & 3 & 3 & 3 & 2 & --}
\levelsixteen{-- & 4 & 4 & 4 & 3 & 3 & 2 & 0}
\levelseventeen{-- & 4 & 4 & 4 & 4 & 3 & 3 & 1}
\leveleighteen{Mass Suggestion & 4 & 4 & 4 & 4 & 4 & 3 & 2}
\levelnineteen{-- & 4 & 4 & 4 & 4 & 4 & 4 & 3}
\leveltwenty{Inspire Courage +4 & 4 & 4 & 4 & 4 & 4 & 4 & 4}
\end{minorcastingclasstable}


\begin{basictable}{Bard Spells Known}{l*{7}{c}}
\textbf{Level} & \textbf{0th} & \textbf{1st} & \textbf{2nd} & \textbf{3rd} & \textbf{4th} & \textbf{5th} & \textbf{6th} \\
1st & 4 & -- & -- & -- & -- & -- & --\\
2nd & 5 & 2\textsuperscript{1} & -- & -- & -- & -- & --\\
3rd & 6 & 3 & -- & -- & -- & -- & --\\
4th & 6 & 3 & 2\textsuperscript{1} & -- & -- & -- & --\\
5th & 6 & 4 & 3 & -- & -- & -- & --\\
6th & 6 & 4 & 3 & -- & -- & -- & --\\
7th & 6 & 4 & 4 & 2\textsuperscript{1} & -- & -- & --\\
8th & 6 & 4 & 4 & 3 & -- & -- & --\\
9th & 6 & 4 & 4 & 3 & -- & -- & --\\
10th & 6 & 4 & 4 & 4 & 2\textsuperscript{1} & -- & --\\
11th & 6 & 4 & 4 & 4 & 3 & -- & --\\
12th & 6 & 4 & 4 & 4 & 3 & -- & --\\
13th & 6 & 4 & 4 & 4 & 4 & 2\textsuperscript{1} & --\\
14th & 6 & 4 & 4 & 4 & 4 & 3 & --\\
15th & 6 & 4 & 4 & 4 & 4 & 3 & --\\
16th & 6 & 5 & 4 & 4 & 4 & 4 & 2\textsuperscript{1}\\
17th & 6 & 5 & 5 & 4 & 4 & 4 & 3\\
18th & 6 & 5 & 5 & 5 & 4 & 4 & 3\\
19th & 6 & 5 & 5 & 5 & 5 & 4 & 4\\
20th & 6 & 5 & 5 & 5 & 5 & 5 & 4\\
\multicolumn{8}{p{8.5cm}}{\textsuperscript{1} Provided the bard has a high enough Charisma score to have a bonus spell of this level.}\\
\end{basictable}

\classfeatures

\textbf{Weapon and Armor Proficiency:} A bard is proficient with all simple weapons, plus the longsword, rapier, sap, short sword, shortbow, and whip. Bards are proficient with light armor and shields (except tower shields). A bard can cast bard spells while wearing light armor without incurring the normal arcane spell failure chance. However, like any other arcane spellcaster, a bard wearing medium or heavy armor or using a shield incurs a chance of arcane spell failure if the spell in question has a somatic component (most do). A multiclass bard still incurs the normal arcane spell failure chance for arcane spells received from other classes.

\textbf{Spells:} A bard casts arcane spells, which are drawn from the bard spell list. He can cast any spell he knows without preparing it ahead of time. Every bard spell has a verbal component (singing, reciting, or music). To learn or cast a spell, a bard must have a Charisma score equal to at least 10 + the spell. The Difficulty Class for a saving throw against a bard's spell is 10 + the spell level + the bard's Charisma modifier.

Like other spellcasters, a bard can cast only a certain number of spells of each spell level per day. His base daily spell allotment is given on Table: The Bard. In addition, he receives bonus spells per day if he has a high Charisma score. When Table: Bard Spells Known indicates that the bard gets 0 spells per day of a given spell level, he gains only the bonus spells he would be entitled to based on his Charisma score for that spell level.

The bard's selection of spells is extremely limited. A bard begins play knowing four 0-level spells of your choice. At most new bard levels, he gains one or more new spells, as indicated on Table: Bard Spells Known. (Unlike spells per day, the number of spells a bard knows is not affected by his Charisma score; the numbers on Table: Bard Spells Known are fixed.)

Upon reaching 5th level, and at every third bard level after that (8th, 11th, and so on), a bard can choose to learn a new spell in place of one he already knows. In effect, the bard "loses" the old spell in exchange for the new one. The new spell's level must be the same as that of the spell being exchanged, and it must be at least two levels lower than the highest-level bard spell the bard can cast. A bard may swap only a single spell at any given level, and must choose whether or not to swap the spell at the same time that he gains new spells known for the level.

As noted above, a bard need not prepare his spells in advance. He can cast any spell he knows at any time, assuming he has not yet used up his allotment of spells per day for the spell's level. 

\textbf{Bardic Knowledge:} A bard may make a special bardic knowledge check with a bonus equal to his bard level + his Intelligence modifier to see whether he knows some relevant information about local notable people, legendary items, or noteworthy places. (If the bard has 5 or more ranks in Knowledge (history), he gains a +2 bonus on this check.)

A successful bardic knowledge check will not reveal the powers of a magic item but may give a hint as to its general function. A bard may not take 10 or take 20 on this check; this sort of knowledge is essentially random. 

\begin{basictable}{Bardic Knowledge Checks}{c p{15cm}}
\textbf{DC} & \textbf{Type of Knowledge} \\
10 & Common, known by at least a substantial minority drinking; common legends of the local population.\\
20 & Uncommon but available, known by only a few people legends.\\
25 & Obscure, known by few, hard to come by.\\
30 & Extremely obscure, known by very few, possibly forgotten by most who once knew it, possibly known only by those who don't understand the significance of the knowledge.\\
\end{basictable}

\textbf{Bardic Music:} Once per day per bard level, a bard can use his song or poetics to produce magical effects on those around him (usually including himself, if desired). While these abilities fall under the category of bardic music and the descriptions discuss singing or playing instruments, they can all be activated by reciting poetry, chanting, singing lyrical songs, singing melodies, whistling, playing an instrument, or playing an instrument in combination with some spoken performance. Each ability requires both a minimum bard level and a minimum number of ranks in the Perform skill to qualify; if a bard does not have the required number of ranks in at least one Perform skill, he does not gain the bardic music ability until he acquires the needed ranks.

Starting a bardic music effect is a standard action. Some bardic music abilities require concentration, which means the bard must take a standard action each round to maintain the ability. Even while using bardic music that doesn't require concentration, a bard cannot cast spells, activate magic items by spell completion (such as scrolls), or activate magic items by magic word (such as wands). Just as for casting a spell with a verbal component, a deaf bard has a 20\% chance to fail when attempting to use bardic music. If he fails, the attempt still counts against his daily limit.

\textit{Countersong (Su):} A bard with 3 or more ranks in a \linkskill{Perform} skill can use his music or poetics to counter magical effects that depend on sound (but not spells that simply have verbal components). Each round of the countersong, he makes a Perform check. Any creature within 30 feet of the bard (including the bard himself ) that is affected by a sonic or language-dependent magical attack may use the bard's Perform check result in place of its saving throw if, after the saving throw is rolled, the Perform check result proves to be higher. If a creature within range of the countersong is already under the effect of a noninstantaneous sonic or language-dependent magical attack, it gains another saving throw against the effect each round it hears the countersong, but it must use the bard's Perform check result for the save. Countersong has no effect against effects that don't allow saves. The bard may keep up the countersong for 10 rounds.

\textit{Fascinate (Sp):} A bard with 3 or more ranks in a Perform skill can use his music or poetics to cause one or more creatures to become \linksec{Fascinated} with him. Each creature to be fascinated must be within 90 feet, able to see and hear the bard, and able to pay attention to him. The bard must also be able to see the creature. The distraction of a nearby combat or other dangers prevents the ability from working. For every three levels a bard attains beyond 1st, he can target one additional creature with a single use of this ability.

To use the ability, a bard makes a Perform check. His check result is the DC for each affected creature's Will save against the effect. If a creature's saving throw succeeds, the bard cannot attempt to fascinate that creature again for 24 hours. If its saving throw fails, the creature sits quietly and listens to the song, taking no other actions, for as long as the bard continues to play and concentrate (up to a maximum of 1 round per bard level). While fascinated, a target takes a -4 penalty on skill checks made as reactions, such as \linkskill{Listen} and \linkskill{Spot} checks. Any potential threat requires the bard to make another Perform check and allows the creature a new saving throw against a DC equal to the new Perform check result.

Any obvious threat, such as someone drawing a weapon, casting a spell, or aiming a ranged weapon at the target, automatically breaks the effect. \textit{Fascinate}is an enchantment (compulsion), mind-affecting ability.

\textit{Inspire Courage (Su):} A bard with 3 or more ranks in a Perform skill can use song or poetics to inspire courage in his allies (including himself), bolstering them against fear and improving their combat abilities. To be affected, an ally must be able to hear the bard sing. The effect lasts for as long as the ally hears the bard sing and for 5 rounds thereafter. An affected ally receives a +1 morale bonus on saving throws against charm and fear effects and a +1 morale bonus on attack and weapon damage rolls. At 8th level, and every six bard levels thereafter, this bonus increases by 1 (+2 at 8th, +3 at 14th, and +4 at 20th). Inspire courage is a mind-affecting ability.

\textit{Inspire Competence (Su):} A bard of 3rd level or higher with 6 or more ranks in a Perform skill can use his music or poetics to help an ally succeed at a task. The ally must be within 30 feet and able to see and hear the bard. The bard must also be able to see the ally.

The ally gets a +2 competence bonus on skill checks with a particular skill as long as he or she continues to hear the bard's music. Certain uses of this ability are infeasible. The effect lasts as long as the bard concentrates, up to a maximum of 2 minutes. A bard can't inspire competence in himself. Inspire competence is a mind-affecting ability.

\textit{Suggestion (Sp):} A bard of 6th level or higher with 9 or more ranks in a Perform skill can make a \linkspell{Suggestion} (as the spell) to a creature that he has already fascinated (see above). Using this ability does not break the bard's concentration on the \textit{fascinate} effect, nor does it allow a second saving throw against the \textit{fascinate} effect.

Making a \textit{suggestion} doesn't count against a bard's daily limit on bardic music performances. A Will saving throw (DC 10 + 1/2 bard's level + bard's Cha modifier) negates the effect. This ability affects only a single creature (but see \textit{mass suggestion}, below). \textit{Suggestion} is an enchantment (compulsion), mind-affecting, language dependent ability.

\textit{Inspire Greatness (Su):} A bard of 9th level or higher with 12 or more ranks in a Perform skill can use music or poetics to inspire greatness in himself or a single willing ally within 30 feet, granting him or her extra fighting capability. For every three levels a bard attains beyond 9th, he can target one additional ally with a single use of this ability (two at 12th level, three at 15th, four at 18th). To inspire greatness, a bard must sing and an ally must hear him sing. The effect lasts for as long as the ally hears the bard sing and for 5 rounds thereafter. A creature inspired with greatness gains 2 bonus Hit Dice (d10s), the commensurate number of temporary hit points (apply the target's Constitution modifier, if any, to these bonus Hit Dice), a +2 competence bonus on attack rolls, and a +1 competence bonus on Fortitude saves. The bonus Hit Dice count as regular Hit Dice for determining the effect of spells that are Hit Dice dependent. Inspire greatness is a mind-affecting ability.

\textit{Song of Freedom (Sp):} A bard of 12th level or higher with 15 or more ranks in a Perform skill can use music or poetics to create an effect equivalent to the \linkspell{Break Enchantment} spell (caster level equals the character's bard level). Using this ability requires 1 minute of uninterrupted concentration and music, and it functions on a single target within 30 feet. A bard can't use \textit{song of freedom} on himself.

\textit{Inspire Heroics (Su):} A bard of 15th level or higher with 18 or more ranks in a Perform skill can use music or poetics to inspire tremendous heroism in himself or a single willing ally within 30 feet. For every three bard levels the character attains beyond 15th, he can inspire heroics in one additional creature. To inspire heroics, a bard must sing and an ally must hear the bard sing for a full round. A creature so inspired gains a +4 morale bonus on saving throws and a +4 dodge bonus to AC. The effect lasts for as long as the ally hears the bard sing and for up to 5 rounds thereafter. Inspire heroics is a mind-affecting ability.

\textit{Mass Suggestion (Sp):} This ability functions like \textit{suggestion}, above, except that a bard of 18th level or higher with 21 or more ranks in a Perform skill can make the \textit{suggestion} simultaneously to any number of creatures that he has already fascinated (see above). \textit{Mass suggestion} is an enchantment (compulsion), mind-affecting, language-dependent ability.
