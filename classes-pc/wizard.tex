%%%%%%%%%%%%%%%%%%%%%%%%%%%%%%%%%%%%%%%%%%%%%%%%%%
\classentry{Wizard}
%%%%%%%%%%%%%%%%%%%%%%%%%%%%%%%%%%%%%%%%%%%%%%%%%%
\tagline{"And as you can see, when I wiggle my left pinky just like this\ldots{} and now your whole house is on fire. Isn't that fantastic?"}

\textbf{Alignment:} A wizard can be of any alignment. Though the science of magic follows many rules, wizards are just as likely to be fickle as not.

\textbf{Races:} Wizards tend to come from places with the civilization to support wizard colleges, so a large number of Wizards come from Human, Elf, or Dwarven lands. However, members of other races can often simply travel to a wizard college if they want to learn the arts. Particularly, many Gnomes often feel the call to become Illusionists.

\textbf{Starting Gold:} 2d6x10 gp (70 gold)

\textbf{Starting Age:} As Wizard.

\textbf{Hit Die:} d4

\textbf{Class Skills:} The wizard's class skills (and the key ability for each skill) are \linkskill{Concentration} (Con), \linkskill{Craft} (Int), \linkskill{Decipher Script} (Int), \linkskill{Knowledge} (Any) (Int), \linkskill{Profession} (Wis), and \linkskill{Spellcraft} (Int)

\textbf{Skills/Level:} 4 + Intelligence Bonus

\poorbab{}
\poorfor{}
\poorref{}
\goodwil{}

\begin{fullcastingclasstable}
\levelone{\multicolumn{1}{p{3cm}}{\raggedright{}Summon Familiar, Scribe Scroll} & 1 & -- & -- & -- & -- & -- & -- & -- & --}
\leveltwo{-- & 2 & -- & -- & -- & -- & -- & -- & -- & --}
\levelthree{-- & 2 & 1 & -- & -- & -- & -- & -- & -- & --}
\levelfour{-- & 3 & 2 & -- & -- & -- & -- & -- & -- & --}
\levelfive{Bonus Feat & 3 & 2 & 1 & -- & -- & -- & -- & -- & --}
\levelsix{-- & 3 & 3 & 2 & -- & -- & -- & -- & -- & --}
\levelseven{-- & 4 & 3 & 2 & 1 & -- & -- & -- & -- & --}
\leveleight{-- & 4 & 3 & 3 & 2 & -- & -- & -- & -- & --}
\levelnine{-- & 4 & 4 & 3 & 2 & 1 & -- & -- & -- & --}
\levelten{Bonus Feat & 4 & 4 & 3 & 3 & 2 & -- & -- & -- & --}
\leveleleven{-- & 4 & 4 & 4 & 3 & 2 & 1 & -- & -- & --}
\leveltwelve{-- & 4 & 4 & 4 & 3 & 3 & 2 & -- & -- & --}
\levelthirteen{-- & 4 & 4 & 4 & 4 & 3 & 2 & 1 & -- & --}
\levelfourteen{-- & 4 & 4 & 4 & 4 & 3 & 3 & 2 & -- & --}
\levelfifteen{Bonus Feat & 4 & 4 & 4 & 4 & 4 & 3 & 2 & 1 & --}
\levelsixteen{-- & 4 & 4 & 4 & 4 & 4 & 3 & 3 & 2 & --}
\levelseventeen{-- & 4 & 4 & 4 & 4 & 4 & 4 & 3 & 2 & 1}
\leveleighteen{-- & 4 & 4 & 4 & 4 & 4 & 4 & 3 & 3 & 2}
\levelnineteen{-- & 4 & 4 & 4 & 4 & 4 & 4 & 4 & 3 & 3}
\leveltwenty{Bonus Feat & 4 & 4 & 4 & 4 & 4 & 4 & 4 & 4 & 4}
\end{fullcastingclasstable}

\classfeatures

\textbf{Weapon and Armor Proficiency:} Wizards are proficient with the club, dagger, heavy crossbow, light crossbow, and quarterstaff, but not with any type of armor or shield. Armor of any type interferes with a wizard's movements, which can cause her spells with somatic components to fail.

\textbf{Spells:} A wizard casts arcane spells which are drawn from the sorcerer/ wizard spell list. A wizard must choose and prepare her spells ahead of time (see below).

To learn, prepare, or cast a spell, the wizard must have an Intelligence score equal to at least 10 + the spell level. The Difficulty Class for a saving throw against a wizard's spell is 10 + the spell level + the wizard's Intelligence modifier.

Like other spellcasters, a wizard can cast only a certain number of spells of each spell level per day. Her base daily spell allotment is given on Table: The Wizard. In addition, she receives bonus spells per day if she has a high Intelligence score.

Unlike a bard or sorcerer, a wizard may know any number of spells. She must choose and prepare her spells ahead of time by getting a good night's sleep and spending 1 hour studying her spellbook. While studying, the wizard decides which spells to prepare.

\textbf{Cantrips:} In addition to their normal allotment of spells per day, a Wizard can prepare a number of 0th level spells, known as "cantrips". A wizard can prepare four cantrips per day, and can cast any prepared cantrip an unlimited number of times.

\textbf{Familiar (Ex):} A wizard can obtain a familiar in exactly the same manner as a sorcerer can. See the sorcerer description and the information on Familiars below for details.

\textbf{Scribe Scroll:} At 1st level, a wizard gains Scribe Scroll as a bonus feat.

\textbf{Bonus Feats:} At 5th, 10th, 15th, and 20th level, a wizard gains a bonus feat. At each such opportunity, she can choose a metamagic feat, an item creation feat, or Spell Mastery. The wizard must still meet all prerequisites for a bonus feat, including caster level minimums.

These bonus feats are in addition to the feat that a character of any class gets from advancing levels. The wizard is not limited to the categories of item creation feats, metamagic feats, or Spell Mastery when choosig these feats.

\textbf{Spellbooks:} A wizard must study her spellbook each day to prepare her spells. She cannot prepare any spell not recorded in her spellbook, except for \linkspell{Read Magic}, which all wizards can prepare from memory.

A wizard begins play with a spellbook containing all 0-level wizard spells (except those from her prohibited school or schools, if any; see School Specialization, below) plus three 1st-level spells of your choice. For each point of Intelligence bonus the wizard has, the spellbook holds one additional 1st-level spell of your choice. At each new wizard level, she gains two new spells of any spell level or levels that she can cast (based on her new wizard level) for her spellbook. At any time, a wizard can also add spells found in other wizards' spellbooks to her own.

%%%%%%%%%%%%%%%%%%%%%%%%%
\subsection{School Specialization}\index{Wizard Specialization}
%%%%%%%%%%%%%%%%%%%%%%%%%

A school is one of eight groupings of spells, each defined by a common theme. If desired, a wizard may specialize in one school of magic (see below). Specialization allows a wizard to cast extra spells from her chosen school, but she then never learns to cast spells from some other schools.

A specialist wizard can prepare one additional spell of her specialty school per spell level each day. She also gains a +2 bonus on \linkskill{Spellcraft} checks to learn the spells of her chosen school.

The wizard must choose whether to specialize and, if she does so, choose her specialty at 1st level. At this time, she must also give up two other schools of magic (unless she chooses to specialize in divination; see below), which become her prohibited schools.

A wizard can never give up divination to fulfill this requirement.

Spells of the prohibited school or schools are not available to the wizard, and she can't even cast such spells from scrolls or fire them from wands. She may not change either her specialization or her prohibited schools later.

The eight schools of arcane magic are abjuration, conjuration, divination, enchantment, evocation, illusion, necromancy, and transmutation.

Spells that do not fall into any of these schools are called universal spells.

\textit{Abjuration:}\index{Abjuration} Spells that protect, block, or banish. An abjuration specialist is called an \gameterm{Abjurer}.

\textit{Conjuration:}\index{Conjuration} Spells that bring creatures or materials to the caster. A conjuration specialist is called a \gameterm{Conjurer}.

\textit{Divination:}\index{Divination} Spells that reveal information. A divination specialist is called a \gameterm{Diviner}. Unlike the other specialists, a diviner must give up only one other school.

\textit{Enchantment:}\index{Enchantment} Spells that imbue the recipient with some property or grant the caster power over another being. An enchantment specialist is called an \gameterm{Enchanter}.

\textit{Evocation:}\index{Evocation} Spells that manipulate energy or create something from nothing. An evocation specialist is called an \gameterm{Evoker}.

\textit{Illusion:}\index{Illusion} Spells that alter perception or create false images. An illusion specialist is called an \gameterm{Illusionist}.

\textit{Necromancy:}\index{Necromancy} Spells that manipulate, create, or destroy life or life force. A necromancy specialist is called a \gameterm{Necromancer}.

\textit{Transmutation:}\index{Transmutation} Spells that transform the recipient physically or change its properties in a more subtle way. A transmutation specialist is called a \gameterm{Transmuter}.

\textit{Universal:} Not a school, but a category for spells that all wizards can learn. A wizard cannot select universal as a specialty school or as a prohibited school. Only a limited number of spells fall into this category.

%%%%%%%%%%%%%%%%%%%%%%%%%
\subsection{Familiars}\index{Familiar}
%%%%%%%%%%%%%%%%%%%%%%%%%

A familiar is a normal animal that gains new powers and becomes a magical beast when summoned to service by a sorcerer or wizard. It retains the appearance, Hit Dice, base attack bonus, base save bonuses, skills, and feats of the normal animal it once was, but it is treated as a magical beast instead of an animal for the purpose of any effect that depends on its type. Only a normal, unmodified animal may become a familiar. An animal companion cannot also function as a familiar.

A familiar also grants special abilities to its master (a sorcerer or wizard), as given on the table below. These special abilities apply only when the master and familiar are within 1 mile of each other.

Levels of different classes that are entitled to familiars stack for the purpose of determining any familiar abilities that depend on the master's level.

\begin{basictable}{Familiar Benefits}{l l}
\textbf{Familiar Type} & \textbf{Master's Benefit} \\
Bat & +3 to \linkskill{Listen} checks.\\
Cat & +3 to \linkskill{Move Silently} checks.\\
Hawk & +3 to \linkskill{Spot} checks in bright light.\\
Lizard & +3 to \linkskill{Climb} checks.\\
Owl & +3 to \linkskill{Spot} checks in shadows.\\
Rat & +2 on Fortitude Saves.\\
Raven\textsuperscript{1} & +3 on \linkskill{Appraise} checks.\\
Snake\textsuperscript{2} & +3 on \linkskill{Bluff} checks.\\
Toad & Gain +3 hit points.\\
Weasel & +2 on Reflex Saves.\\
\multicolumn{2}{l}{\textsuperscript{1} A raven familiar can speak one language of the master's choice as a Supernatural ability.}\\
\multicolumn{2}{l}{\textsuperscript{2} Tiny Viper.}\\
\end{basictable}

\textbf{Familiar Basics:} Use the basic statistics for a creature of the familiar's kind, but make the following changes:

\textit{Hit Dice:} For the purpose of effects related to number of Hit Dice, use the master's character level or the familiar's normal HD total, whichever is higher.

\textit{Hit Points:} The familiar has one-half the master's total hit points (not including temporary hit points), rounded down, regardless of its actual Hit Dice.

\textit{Attacks:} Use the master's base attack bonus, as calculated from all his classes. Use the familiar's Dexterity or Strength modifier, whichever is greater, to get the familiar's melee attack bonus with natural weapons.

Damage equals that of a normal creature of the familiar's kind.

\textit{Saving Throws:} For each saving throw, use either the familiar's base save bonus (Fortitude +2, Reflex +2, Will +0) or the master's (as calculated from all his classes), whichever is better. The familiar uses its own ability modifiers to saves, and it doesn't share any of the other bonuses that the master might have on saves.

\textit{Skills:} For each skill in which either the master or the familiar has ranks, use either the normal skill ranks for an animal of that type or the master's skill ranks, whichever are better. In either case, the familiar uses its own ability modifiers. Regardless of a familiar's total skill modifiers, some skills may remain beyond the familiar's ability to use.

\textbf{Familiar Ability Descriptions:} All familiars have special abilities (or impart abilities to their masters) depending on the master's combined level in classes that grant familiars, as shown on the table below. The abilities given on the table are cumulative. 

\textit{Natural Armor Adj.:} The number noted here is an improvement to the familiar's existing natural armor bonus.

\textit{Int:} The familiar's Intelligence score.

\textit{Alertness (Ex):} While a familiar is within arm's reach, the master gains the Alertness feat.

\textit{Improved Evasion (Ex):} When subjected to an attack that normally allows a Reflex saving throw for half damage, a familiar takes no damage if it makes a successful saving throw and half damage even if the saving throw fails.

\textit{Share Spells:} At the master's option, he may have any spell (but not any spell-like ability) he casts on himself also affect his familiar. The familiar must be within 5 feet at the time of casting to receive the benefit.

If the spell or effect has a duration other than instantaneous, it stops affecting the familiar if it moves farther than 5 feet away and will not affect the familiar again even if it returns to the master before the duration expires. Additionally, the master may cast a spell with a target of "You" on his familiar (as a touch range spell) instead of on himself.

A master and his familiar can share spells even if the spells normally do not affect creatures of the familiar's type (magical beast).

\textit{Empathic Link (Su):} The master has an empathic link with his familiar out to a distance of up to 1 mile. The master cannot see through the familiar's eyes, but they can communicate empathically. Because of the limited nature of the link, only general emotional content can be communicated.

Because of this empathic link, the master has the same connection to an item or place that his familiar does.

\textit{Deliver Touch Spells (Su):} If the master is 3rd level or higher, a familiar can deliver touch spells for him. If the master and the familiar are in contact at the time the master casts a touch spell, he can designate his familiar as the "toucher." The familiar can then deliver the touch spell just as the master could. As usual, if the master casts another spell before the touch is delivered, the touch spell dissipates.

\textit{Speak with Master (Ex):} If the master is 5th level or higher, a familiar and the master can communicate verbally as if they were using a common language. Other creatures do not understand the communication without magical help.

\textit{Speak with Animals of Its Kind (Ex):} If the master is 7th level or higher, a familiar can communicate with animals of approximately the same kind as itself (including dire varieties): bats with bats, rats with rodents, cats with felines, hawks and owls and ravens with birds, lizards and snakes with reptiles, toads with amphibians, weasels with similar creatures (weasels, minks, polecats, ermines, skunks, wolverines, and badgers). Such communication is limited by the intelligence of the conversing creatures.

\textit{Spell Resistance (Ex):} If the master is 11th level or higher, a familiar gains spell resistance equal to the master's level + 5. To affect the familiar with a spell, another spellcaster must get a result on a caster level check (1d20 + caster level) that equals or exceeds the familiar's spell resistance.

\textit{Scry on Familiar (Sp):} If the master is 13th level or higher, he may scry on his familiar (as if casting the \textit{scrying }spell) once per day.

\begin{basictable}{Familiar Progression}{l c c l}
\textbf{Level} & \textbf{Nat Armor} & \textbf{Int} & \textbf{Special}\\
1st-2nd & +1 & 6 & Alertness, Improved Evasion, Share Spells, Empathic Link\\
3rd-4th & +2 & 7 & Deliver Touch Spells\\
5th-6th & +3 & 8 & Speak with Master\\
7th-8th & +4 & 9 & Speak with animals of its kind.\\
9th-10h & +5 & 10 & --\\
11th-12th & +6 & 11 & Spell Resistance\\
13th-14th & +7 & 12 & Scry on familiar\\
15th-16th & +8 & 13 & --\\
17th-18th & +9 & 14 & --\\
19th-20th & +10 & 15 & --\\
\end{basictable}
