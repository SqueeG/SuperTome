%%%%%%%%%%%%%%%%%%%%%%%%%%%%%%%%%%%%%%%%%%%%%%%%%%
\section{The Three Economies}
%%%%%%%%%%%%%%%%%%%%%%%%%%%%%%%%%%%%%%%%%%%%%%%%%%
\tagline{"100 pounds of gold for a house? How does anyone make rent without a wheelbarrow?"}

%%%%%%%%%%%%%%%%%%%%%%%%%
\subsection{The Turnip Economy}
%%%%%%%%%%%%%%%%%%%%%%%%%

turnipz

%%%%%%%%%%%%%%%%%%%%%%%%%
\subsection{The Gold Economy}
%%%%%%%%%%%%%%%%%%%%%%%%%

bitcoinz

%%%
\subsubsection{Trade Goods}
%%%

%%%
\subsubsection{Gems}
%%%

%%%
\subsubsection{Darkwood}
%%%

%%%
\subsubsection{Mithral}
%%%

%%%
\subsubsection{Adamantine}
%%%

%%%%%%%%%%%%%%%%%%%%%%%%%
\subsection{The Wish Economy}
%%%%%%%%%%%%%%%%%%%%%%%%%

Powerful people have access to a spell called \linkspell{Wish}, and it can generate a magical item worth 15,000gp or less in a split second. It can also generate most other things you care to name worth less than 15,000gp as well, including all the special materials talked about above. The point is that there are things that it can \textit{not} generate out of thin air. These are the things that powerful people care about. Everything else is just chump change to them.

Now, in addition to Wishing for the things you want, you could go out and build them too. Except that to build things you need materials, and to build things you can't wish for, you also need materials that you can't wish for. It's a real pain like that. These special materials are known as "wish economy materials". Some examples are given here, but it's easy enough to invent your own.

%%%
\subsubsection{Souls}
%%%

The souls of powerful creatures can be trapped in gems, and the soul trade is brisk on the outer planes. Once a soul is in a gem, the gem itself is of little to no value, but the soul goes for 100 gp times the square of the CR of the creature whose soul is trapped.

%%%
\subsubsection{Concentration}
%%%

Ideas take form on the outer planes, and really pernicious or stellar ideas can be so powerful that they take a while to form. In the before-time, they can be found as an amber-like substance that is extremely valued on Mechanus, and by extension every single other outer plane as well. Concentration is actually made out of ideas, and while it looks like a solid object it is actually a liquid that flows so slowly that you could watch it for a year and only a construct could tell you how far the flow had taken it. A pound of concentration goes for 50,000 gp to an interested party.

%%%
\subsubsection{Hope}
%%%

Hope is funny stuff, it has lots of inertia, but those who carry it are not weighed down in the least. It has mass, but not weight. Even the smallest piece of Hope sheds light like a \linkspell{Daylight} spell (effective spell level 7). Hope is measured in kilograms rather than pounds, and a kilo of Hope goes for 100,000 gp to those who want it.

%%%
\subsubsection{Raw Chaos}
%%%

The plane of Limbo is filled with possibility and change. Usually this manifests as a continuous creation and destruction that is awe inspiring and terrifying at the same time. Sometimes, for whatever reason, this possibility doesn't become anything, and just stays as Raw Chaos. Raw Chaos can have any dimensions and any amount of mass, but from a practical standpoint you either have it or you don't. If you have Raw Chaos and someone else doesn't you can give it to them, and it is generally considered good form for them to give you magical items or planar currency worth 200,000 gp in exchange.

%%%%%%%%%%%%%%%%%%%%%%%%%
\subsection{Getting Paid In Favors}
%%%%%%%%%%%%%%%%%%%%%%%%%

foovers

%%%%%%%%%%%%%%%%%%%%%%%%%
%%%%%%%%%%%%%%%%%%%%%%%%%
%%%%%%%%%%%%%%%%%%%%%%%%%

Since time immemorial, past editions have used the "gold piece" as the primary currency. It is apparently a chunk of reasonably pure gold of vaguely standardized weight that people use fairly interchangeably in different cities populated by different species. In the bad old days, each gold coin was a tenth of a pound, which was hilarious and inane. In the current edition, each gold piece is a fiftieth of a pound. That's 3.43 gp to the Troy Ounce, which means that in the modern economy, each gp is about \$171 worth of gold. Obviously, gold is significantly more common in the game than it is on Earth, gold is also undervalued because its status as a currency standard drives it out of industrial uses and causes inflation. Further, populations are orders of magnitude smaller than they are in the real world, so the gold per person is higher even with the same amount of gold. So the gold piece is massively less valuable in these economies than it would be in Earth's economies.

Nonetheless, things are really expensive, and the high price in gold means that there's a distinct limitation of how much wealth can be transported by any means available. The economies of currency transaction are actually so unfavorable that currency as we understand the term does not exist. Things don't have prices or costs -- all transactions are conducted in barter and a common medium of exchange is heavy lumps of precious metal.

%%%%%%%%%%%%%%%%%%%%%%%%%
\subsection{Wartime Economies Make for Shortages}
%%%%%%%%%%%%%%%%%%%%%%%%%

Many people wonder why a masterwork dagger goes for more than its weight in gold. That's a pretty valid question to ask; certainly I'm not going to attempt to justify the 600 gp price tag on a masterwork walking stick -- that's just an example of simplistic game mechanics run amok. But to an \textit{extent} the crazy prices can be justified by the fact that every settlement on every plane is on a war footing \textit{all the time}. The idea that Peace is somehow a natural state is a fairly recent one, and based on the frequency of wars all over the world -- it's obviously just wishful thinking anyway. War is the default position of every major economy in the world, and that means that weapons have an immediate, and desperate, clientele. Iron is still relatively cheap, because you can't kill people with it \textit{right now}, but actual weapons and armor are crazy expensive.

That doesn't explain the fact that the game charges you over a quarter Oz. of gold just to get a backpack, and it doesn't explain the fact that the markup on masterworking a buckler is the same as the markup on masterworking a breastplate -- that's just a game simplification that makes no real-world sense. But it's a start.

%%%%%%%%%%%%%%%%%%%%%%%%%
\subsection{Coins are Big and Heavy}
%%%%%%%%%%%%%%%%%%%%%%%%%
\tagline{How many boards could the Mongols hoard if the Mongol hordes got bored?"}

From the standpoint of the adventurer, the primary difficulty of the gold currency system is that the lack of a coherent banking and paper currency system means that there are profound limits to what you could possibly purchase even with platinum. But the currency system hurts on the other end as well. Untrained labor gets a silverpiece a \textit{week}. That's 500 copper coins a year, which means that no matter how cheap things are they can only make one purchase a day most of the time. That's pretty stifling to the economy, in that however much gets produced, no one can \textit{buy} it. Demand, from the economics standpoint, is strangled to the point where large production outputs don't even matter (remember that in economics \textit{Demand} doesn't mean "what people want", it means "what people are willing \textit{and able} to pay for", so if the average person only has 500 discreet pieces of currency per year, that puts an absolute cap on economic demand, even though the people are of course both needy and greedy enough to want anything you happen to produce).

What's worse, those coins are \textit{heavy}. For our next demonstration, reach into your change drawer and fish out nine pennies. That's a decent lump in your pocket, neh? That's about \textbf{one} copper piece. Gold pieces are smaller (less than half the size, actually), but weigh the same. Metal currency, therefore, is more like a Monopoly playing piece than it is like a modern or ancient coin. There's no reason to even believe these things are \textit{round}, people are seriously marching around gold hats and silver dogs as the basic medium of exchange.

Now, you may ask yourself why these coins are so titanic compared to real coins. The answer is because having piles of coins is awesome. Dragons are supposed to \textit{sleep} on that stuff, and that requires big piles of coins. Consider my own mattress, which is a "twin-size" (pretty reasonable for a single medium-size creature) and nearly .2 cubic meters. If it was made out of gold, it would be about 3.9 tonnes. That's about eighty-six hundred pounds, and even with the ginormous coins this game has, that's four hundred and thirty \textbf{thousand} gold pieces. In previous editions, that sort of thing was simply accepted and very powerful dragons really did have the millions of gold pieces -- which was actually fine. Since third edition, they've been trying to make gold actually equal character power, and the result has been that dragon hoards are\ldots{} really small. None of this "We need to get a wagon team to haul it all away", no. In 3rd edition, hoard sizes have become manageable, even ridiculously tiny. When a 6th level party defeats a powerful and wealthy monster, they can expect to find\ldots{} nearly a liter of gold. That is, the treasure "hoard" of that evil dragon you defeated will actually fit into an Evian bottle.

There are two ways to handle this:

1. Live with the fact that treasures are small and unexciting.

2. Live with the fact that characters who grab a realistic dragon's hoard become filthy stinking rich and this fundamentally changes the way they interact with society.

But once you accept that the realities of the \textit{wish} based economy, you actually \textbf{don't} have to live with characters unbalancing the game once they find a real mattress filled with gold. That's not even a problem once characters are no longer excited by a +2 Enhancement bonus to a stat or a +3 enhancement bonus to Armor. Which means somewhere between 9th and 13th level it's perfectly fine for players to find actual money without unbalancing the \textit{game}. Really, you can stop worrying about it.

%%%%%%%%%%%%%%%%%%%%%%%%%
\subsection{Bad Money Drives Out Good: The Penalties of Paper}
%%%%%%%%%%%%%%%%%%%%%%%%%

People from the modern world are generally pretty perplexed by this idea of handing back and forth actual metal as a medium of exchange. It is an undeniable truth in our lives that the idea of currency is just that: an \textit{idea}. As long as whatever I'm trading for goods and services can be traded for goods and services, it doesn't actually matter if the exchange commodity has any ascribed intrinsic worth. Paper descriptions of value or even ephemeral electronic representations are not only adequate, they're \textit{convenient}. But more than that, using valuable commodities as a medium of exchange inhibits the growth of the economy. As long as a certain portion of the wealth is locked up in currency, the economy is strangled coming and going: not only is there a completely arbitrary limit on how many goods and services can be exchanged (the gold supply), but there is also a limit on the kinds of industry and artistic expression that can occur (in that if you use gold for anything \textit{but} currency you're actually shrinking the money supply and producing negative GDP).

So\ldots{} you're going to solve that by instituting a paper-based exchange system where initially the paper is exchangeable for gold and that eventually gets phased out when the Plebes realize that handing actual gold back and forth is inconvenient and dumb, right? Wrong. Remember that this is the Iron Age, and people haven't invented Nationalism yet. The cornerstone of the Greenback currency is a belief in the nation that prints it -- and nations simply don't exist. You've got empires, and you've got kingdoms, and you've got tribes, and you've got unincorporated villages\ldots{} and that's it as far as civilization goes. When you look at a map and a colored region has a name on it, that's the name of the \textit{region}. Possibly it's even the name of some guy \textit{in} the region. The point is, that it's not a country in the modern sense of the word, so if some new guy walks in who's bad enough the next cartographer will put \textit{his} name on the region instead.

And that means that "The Full Faith and Credit of the Kingdom of Daxall" is worth precisely \textit{nothing}. And while King Daxall can, through force of arms, take all the gold away from all the peasants and get them to trade pieces of paper for goods and services in its place -- noone will actually \textit{believe} that the paper is currency. They're literally trading promises by King Daxall that he'll let them have their money back if they leave town. And since the serfs can't even leave town, even that promise is meaningless to them. A serf accepts paper for goods and services only because he'll be beheaded if he doesn't. The black market value of these pieces of paper is pretty close to zero. Worse, nearby governments will see this as a blatant attempt to sequester all the gold in King Daxall's pants and will probably declare war (in addition to the fact that noone outside the reach of King Daxall's pikemen will accept Daxall Dollars).

%%%%%%%%%%%%%%%%%%%%%%%%%
\subsection{Powerful Creatures Have a Powerful Economy}
%%%%%%%%%%%%%%%%%%%%%%%%%

The amount of gold it takes to get anywhere as a land lord is \textit{very large}. The question that arises then, is why awesome architecture exists \textit{at all}. It's a valid question, the listed costs to put things like pit traps and thrones made of bone into your dungeon are stupendously large and actual magical swag can be made available for much less than that. The answer is that:

1. People don't actually pay all that gold to have their homes remodeled (see the peonomicon below).

2. Powerful artificers and adventurers don't even \textit{want} your gold. If something has a value of 100,000 gold pieces, it can't be purchased with gold pieces at all -- because that's an actual ton of gold that you'd have to plop over the counter and the merchant you're dealing with won't take your money even if you have it.

Here we're going to be focusing in on 

\begin{itemize*}
\item Gems
\item Souls
\item Concentration
\item Hope
\item Raw Chaos
\end{itemize*}

%%%
\subsubsection{Gems: Truth or Dare}
%%%

Gems are, to the vast majority of participants in the economy, pretty much worthless. A 500 gp diamond is pretty much the same as a gold piece to someone who intends to purchase things with a value of 1 gp or less. And of course, there are a lot more individuals out there who will stab a peasant in the face for a diamond than a gold piece. So why does anyone care?

Well, two reasons: the first is the obvious one that gold is extremely \textit{limited} in what it can possibly purchase. A +2 sword is worth your weight in gold. Not \textit{its} weight in gold, \textit{your} weight in gold. It seriously costs over 166 pounds of gold, and that's just not reasonable for most people to put into their pockets. So people interacting with even the shallow end of the magic trade \textit{need} there to be some crazily expensive items that have no purpose save to look pretty and be exchangeable for other stuff. But unlike our world gems actually have real value as well: as the fuel for powerful magics.

On Earth, the only reason that a diamond is expensive is because there's an international organization called DeBeers that seriously has actual assassins that will shoot you in the face if you try to sell diamonds for less than the price they've determined that they're supposed to be sold for. This game doesn't have that kind of armed monopoly to maintain gem prices, but it does have the fact that people continuously use up gems for spells like \textit{raise dead} and item creation and the like. So the fact that you can use ruby dust to make \textit{continual flames} that you can turn around and sell as Everburning Torches means that ruby dust will continue to have value as long as people value light.

The rules actually only go into the spell component uses of a handful of gems, but rest assured that all the rest are similarly useful when we get into the ephemerals of item creation. A lot of those "components" that cost piles of thousands of gold pieces are actually just piles of gems. Onyx keeps its value based on the needs of necromancers, but amethyst is just as needed to bind illusion magic into a cloak. The exchange rate between gems and magic items is in no danger of going \textit{anywhere}. Minor magic items and gems are traded avidly by shopkeepers, adventurers, and even powerful outsiders and wizards.

But even so, gems can be simply acquired by the very powerful. The realities of the \textit{wish} based economy ensure that gems can simply be obtained in large numbers by anyone who \textit{really} cares enough to dedicate a conjured earth elemental to collecting them. Magical items that cannot be created with the application of spells (that is, magic items valued at more than 15,000 gp) cannot be purchased on the open market with mundane currency, not even gems. That isn't to say that you can't cheat a goblin out of a staff of power with some shiny rocks, you totally can (heck, you could also \textit{stab the goblin in the face} and take that staff of power), but doing so is not considered a "fair trade" and requires a bluff check on your part.

In addition, many game worlds posit the existence of \textit{magic gems}, which can be used to make magic items, increase personal power, make a snazzy grill with the bottom row made of gold, and all kinds of stuff. In addition to getting hot women to ask you to smile, these magical gems are \textit{magical} and are actually considered fair exchange in the near-epic economy. You can't wish for Eberron Dragonshards or Planescape Planar Pearls, so those things have real value to Efreet and other creatures participating in the Big Pond. Rules for using magic gems appear in the Tome of Tiamat.

%%%%%%%%%%%%%%%%%%%%%%%%%
\subsection{Bringing the World out of the Dark Ages}
%%%%%%%%%%%%%%%%%%%%%%%%%

It is historical fact that you can take a ridiculous and crumbling imperium with serfs and horse-drawn carts managed by a tyrannical and squabbling aristocracy and boot strap it into being a technologically sophisticated global power that can win the space race and such in a single generation even while being invaded by an evil and genocidal empire. The people at the top don't even need to be nice \textit{or sane}, they just have to understand that economics is an entirely voodoo science, and the limits of production can be broken by thousands of percentage points by getting everyone to buy on credit, work on projects that people looking at the big picture tell them to work on, continuously invest in productive capital, and believe in the future.

Right. That's called Communism, and it ends the dark ages immediately even if it isn't run well. Presumably if it was being run by Paladins who actually \textit{radiate goodness} and Wizards who are inhumanly intelligent and can cast powerful divinations to determine projected needs and goods could be distributed to the masses with teleportals -- it would work substantially better. That sort of thing is not outside the capabilities of your characters. It's not outside the capabilities of the people in the village your characters are saving from gnollish invasion. It's not even technically complicated. But it isn't done.

Partly it isn't done because that's just not the game we're playing. While it is true that you \textit{can} fix the world's ills in a much more tangible fashion by industrializing the production of grain and arranging a non-gold based distribution system such that staple food stuffs are available to all, thereby freeing up potential productive labor for use in blah blah blah\ldots{} the fact is that to a very real degree we play this game because telling stories about slaying evil necromancers and swinging on chandeliers is \textit{awesome}. But the other reason is that the society really isn't ready for a modern or futuristic social setup. No one is going to understand how they are supposed to interact with Socialism, Capitalism, or Fascism. Things are Feudal and people \textit{understand} that. Wealth is exchanged for goods and services on the grounds that people on both sides of the exchange aren't sure that they would win the resulting combat if they tried to take the goods or wealth by force of arms.

Rome had steam engines. Actual difference engines that propelled a metal device with the power of a combustion reaction through the medium of the expansion of heated water. Really. They never built rail roads because slaves were \textit{cheaper than donkeys} and the concept of investing in labor saving devices was preposterous. The idea of having an economy based around trust in the government and labor/wealth equivalences is similarly preposterous. It's not that the idea wouldn't work, it's that every man, woman, and child in society would simply laugh you out of the room if you tried to explain it to them. 
